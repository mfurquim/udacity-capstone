\documentclass[12pt,openright,hidelinks,twoside,a4paper,english,french,spanish]{abntex2}

\usepackage{cmap}
\usepackage{lmodern}
\usepackage[T1]{fontenc}
\usepackage[utf8]{inputenc}
\usepackage{lastpage}
\usepackage{indentfirst}
\usepackage[table]{xcolor}
\usepackage{graphicx}
\usepackage{units}
\usepackage[brazilian,hyperpageref]{backref}
\usepackage[alf]{abntex2cite}
\usepackage{bold-extra}
\usepackage{eso-pic}
\usepackage[procnames]{listings}
\usepackage{enumerate}
\usepackage[section]{placeins}

\usepackage{amsthm}
\usepackage{amssymb, amsmath, MnSymbol}
\usepackage{multirow}
\usepackage{longtable}
\usepackage{rotating}
\usepackage{textcomp}
\usepackage{nicefrac}
\usepackage{tikz}
\usetikzlibrary{patterns}
\usepackage{subcaption}
\usepackage{pdfpages}
\usepackage{colortbl}
\usepackage{pgf}
\usepackage{fancyvrb}
\usepackage{placeins}
\usepackage{chngcntr}

\definecolor{blue}{RGB}{41,5,195}
\makeatletter
\hypersetup{
     	%pagebackref=true,
		pdftitle={\@title},
		pdfauthor={\@author},
    	pdfsubject={\imprimirpreambulo},
	    pdfcreator={LaTeX with abnTeX2},
		pdfkeywords={abnt}{latex}{abntex}{abntex2}{trabalho acadêmico},
		colorlinks=true,       		% false: boxed links; true: colored links
    	linkcolor=blue,          	% color of internal links
    	citecolor=blue,        		% color of links to bibliography
    	filecolor=magenta,      		% color of file links
		urlcolor=blue,
		bookmarksdepth=4
}
\makeatother
\setlength{\parindent}{1.3cm}
\setlength{\parskip}{0.2cm}
\makeindex


\begin{document}

%\frenchspacing

\hypertarget{capstone-proposal}{%
\section*{Capstone Proposal: Stock Recommendation}\label{capstone-proposal}}

People who operates in day trade have to analyze the previous days and
recognize the stock movement patterns. There is a certain limit to how
much attention humans can give to tasks. Opening multiple computer
windows to operate on different companies' assets can be effortful.

Whereas, Machine Learning models are capable of recognizing patterns and
make decisions faster. Furthermore, it can be scaled to cover more stock
options.

The proposal for this capstone is an app which will recommend stock
operations to be concluded in the same day.

\hypertarget{data}{%
\subsection*{Data}\label{data}}

The dataset used is the \emph{Huge Stock Price Data: Intraday Minute
Bar} by \href{https://www.kaggle.com/arashnic}{M\"{o}bius} which can be
found at
\href{https://www.kaggle.com/datasets/arashnic/stock-data-intraday-minute-bar}{kaggle}
and
\href{https://github.com/FutureSharks/financial-data/tree/master/pyfinancialdata}{github}.

The data used to train the model will be the minute-by-minute data of
stocks. Among many others, the stocks can be:

\begin{lstlisting}
$ tree -L 2 data/stocks
data/stocks/
\-- histdata/
    |-- ETXEUR/
    |-- GRXEUR/
    |-- JPXJPY/
    |-- SPXUSD/
    \-- README.md
\end{lstlisting}

Here is an example of the EUR\_USD:

\begin{lstlisting}[language=Python]
import pyfinancialdata
data = pyfinancialdata.get(
    provider="oanda",
	instrument="EUR_USD",
	year=2017,
)
data.tail(3)
                       close     high      low     open  volume    price
date
2017-12-29 21:57:00  1.20045  1.20071  1.20004  1.20018      50  1.20045
2017-12-29 21:58:00  1.20041  1.20041  1.20041  1.20041       1  1.20041
2017-12-29 21:59:00  1.20039  1.20039  1.19970  1.20036      14  1.20039
\end{lstlisting}

The data will be explored and cleaned if necessary. Then, it will be
prepared and transformed to provide the model.

\hypertarget{algorithms-and-models}{%
\subsection*{Algorithms and models}\label{algorithms-and-models}}

A couple of algorithms will be defined and the models will be trained
with different hyperparameters. The models will be compared between
themselves according to accuracy and efficiency. The accuracy will be
calculated by the root mean square error of the prediction and the real
data. The efficiency will be measured by calculating the possible profit
by the end of the day.

One model will be chosen to be deployed to production. An api will be
developed to access the model and calculate the recommendation. The api
will then calculate the amount and whether to buy or sell the stock in
order to have profit.

A report will be written structured as the examples given at
\href{https://github.com/udacity/machine-learning/tree/master/projects/capstone}{udacity/machine-learning}

\hypertarget{reference}{%
\subsection*{Reference}\label{reference}}

The materials that will be used as reference are the following:
\begin{itemize}
	\item[Course] \href{https://www.khanacademy.org/economics-finance-domain/core-finance/stock-and-bonds}{Unit: Stocks and bonds} from Khan Academy
	\item[Course] \href{https://classroom.udacity.com/courses/ud501}{Machine Learning for Trading} from Udacity
	\item[Book] Machine Learning for Algorithmic Trading by Stefan Jansen
	\item[Book] \href{https://home.tpq.io/books/py4fi/}{Python for Finance} by Yves Hilpisch
\end{itemize}

\end{document}
